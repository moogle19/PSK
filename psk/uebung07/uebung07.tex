\documentclass[a4paper, 12pt]{article}
\usepackage[utf8]{inputenc}
\usepackage{fullpage}
\usepackage[ngerman]{babel}
\usepackage{amsmath}

\usepackage{minted}

\title{7. Übungsblatt zu Programmiersprachenkonzepte}
\author{Manuel Schwarz, Michael Stypa}

\begin{document}

\maketitle

\subsection*{Aufgabe 7.1}

\subsubsection*{Scope}
\begin{itemize}
  \item Anweisungsbreich, in dem eine Variable sichtbar ist (globale vs.
        lokale Variable)
  \item static scope $\rightarrow$ Scope zur Compilezeit
  \item dynamic scope $\rightarrow$ Scope erst zur Laufzeit bestimmbar
\end{itemize}

\subsubsection*{Lebenszeit}
\begin{itemize}
  \item $\neq$ scope
  \item Zeitraum in dem eine Variable Speicherplatz belegt
  \item Lebensdauer lokaler Variablen: vom Funktionsaufruf bis Verlassen der
        Funktion
  \item Lebensdauer globaler Variablen: ganze Programmdauer
  \item statische lokale Variablen (z.B. C): Lebensdauer wie globale,
        Gültigkeitsbereich: nur lokale Funktion
\end{itemize}

\subsection*{Aufgabe 7.2}

\subsubsection*{Statisches Binden}
\begin{itemize}
  \item Binden zur Compilezeit
  \item bleibt bis zum Programmende bestehen
  \item alle benötigten Daten(-typen) werden mit in die ausführbare Datei
        gebunden
  \item beschleunigt die Ausführung des Programms, führt aber zu mehr 
        Platzverbrauch
\end{itemize}

\subsubsection*{Dynamisches Binden}
\begin{itemize}
  \item Binden zur Laufzeit
  \item kann sich während der Programmausführung ändern
  \item Daten werden erst bei Bedarf (zur Laufzeit) nachgeladen
  \item weniger Platzbedarf, aber Typ-Bindung und -Überprüfung zur Laufzeit
        kostet viel Rechenaufwand
  \item Sprachen mit dynamischer Typbindung haben i.d.R. Interpreter
\end{itemize}

\subsection*{Aufgabe 7.3}

Bezeichner bestehen aus 1 bis 127 signifikanten Characters.

\inputminted{pascal}{aufg3.pas}

\subsection*{Aufgabe 7.4}
\inputminted{pascal}{aufg4.pas}

% \begin{enumerate}
%  \item \mintinline{pascal}@x1 := odd(x2) < (sqrt(x3) >= 3.8)@\\
%        \mintinline{pascal}@x1:boolean; x2:integer; x3:real;@
%  \item \mintinline{pascal}@x4 := pred(x5);@
%        \mintinline{pascal}@@
%  \item \mintinline{pascal}@x6 := chr(sqr(abs(x7 * x8)))<>’H’;@
%        \mintinline{pascal}@@
%  \item \mintinline{pascal}@x9 := ord(x10) * x11 / x12@
%        \mintinline{pascal}@@
% \end{enumerate}



\subsection*{Aufgabe 7.5}
\inputminted[linenos]{pascal}{aufg5.pas}
Korrektur:
%TODO learn how to set language globally
\begin{itemize}
  \item[Zeile 1]  Klammern weg (oder Parameter rein)
  \item[Zeile 3]  In dieser Reihenfolge startet die
                    \mintinline{pascal}@for@-Schleife nicht
  \item[Zeile 4]  \mintinline{pascal}@7.5e+0@
  \item[Zeile 5]  Deklaration von \mintinline{pascal}@pi@ vor der von
                    \mintinline{pascal}@minuspi@\\
                    \mintinline{pascal}@pi@ ist Systemvariable und wird
                    überschrieben
  \item[Zeile 9]  Variablen- versucht Konstatendeklaration zu überschreiben.
  \item[Zeile 11] \mintinline{pascal}@boolean@
  \item[Zeile 12] \mintinline{pascal}@factor@
  \item[Zeile 17] Körper mit \mintinline{pascal}@begin@ und
                    \mintinline{pascal}@end@ für die
                    \mintinline{pascal}@while@-Schleife
  \item[Zeile 18] \mintinline{pascal}@/@ für real
  \item[Zeile 19] kein \mintinline{pascal}@;@ vor \mintinline{pascal}@else@
\end{itemize}

\end{document}
