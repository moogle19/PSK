\documentclass[a4paper,12pt]{article}
\usepackage[utf8]{inputenc}
\usepackage{fullpage}
\usepackage{ngerman}
\usepackage{amsmath}
\usepackage{amssymb}

\usepackage{enumerate}
\usepackage{listings}

\lstset{language=Java}
\lstset{identifierstyle=\ttfamily}

\title{2. Übungsblatt zu Programmiersprachenkonzepte}
\author{Manuel Schwarz, Michael Stypa}

\begin{document}

\maketitle

\subsection*{Aufgabe 1}
\begin{enumerate}[a)]
  \item partiell, Definitionsbereich:
        $\begin{aligned} x \in \mathbb{R} \backslash \{0\} \end{aligned}$ 
  \item total, Definitionsbereich:
        $\begin{aligned} x \in \mathbb{R} \end{aligned}$
  \item partiell, Definitionsbereich:
        $\begin{aligned} x \in \{-2k | k \in \mathbb{N}\cup\{0\}\}
         \end{aligned}$
\end{enumerate}

\bigskip

\subsection*{Aufgabe 2}
Der Java-Compiler wirft trotz des ungewissen Terminierung des Programms weder
eine Warnung noch einen Fehler.
\lstinputlisting{userwhile.java}
\bigskip

\newpage

\subsection*{Aufgabe 3}
Die Turingmaschine geht solange nach rechts, bis die Sequenz
\textquotedblleft01110\textquotedblright gefunden wurde
und fährt dann zurück auf die erste 1.

\vspace{5mm}

\begin{center}
\begin{tabular}{|c|c|c|c|c|c|}\hline

akt. Zustand & akt. Zeichen & $ \longrightarrow $  & neues Zeichen 
& neuer Zust. & Lese-, Schreibkopf\\\hline\hline
  
$ s_{0}$ & 1 & $ \longrightarrow $ & 1 &$ s_{0} $& r \\\hline
$ s_{0}$ & 0 &$ \longrightarrow$ & 0 &$ s_{1}$ & r \\\hline
$ s_{1}$ & 0 &$ \longrightarrow$ & 0 &$ s_{1}$ & r \\\hline
$ s_{1}$ & 1 &$ \longrightarrow$ & 1 &$ s_{2}$ & r \\\hline
$ s_{2}$ & 0 &$ \longrightarrow$ & 0 &$ s_{1}$ & r \\\hline
$ s_{2}$ & 1 &$ \longrightarrow$ & 1 &$ s_{3}$ & r \\\hline
$ s_{3}$ & 0 &$ \longrightarrow$ & 0 &$ s_{1}$ & r \\\hline
$ s_{3}$ & 1 &$ \longrightarrow$ & 1 &$ s_{4}$ & r \\\hline
$ s_{4}$ & 0 &$ \longrightarrow$ & 0 &$ s_{5}$ & l \\\hline
$ s_{4}$ & 1 &$ \longrightarrow$ & 1 &$ s_{0}$ & r \\\hline
$ s_{5}$ & 1 &$ \longrightarrow$ & 1 &$ s_{5}$ & l \\\hline
$ s_{5}$ & 0 &$ \longrightarrow$ & 0 &$ s_{6}$ & r \\\hline
$ s_{6}$ & 1 &$ \longrightarrow$ & 1 &$ s_{6}$ & - \\\hline

\end{tabular}
\end{center}

\bigskip

\subsection*{Aufgabe 4}
Beim Programmstart wird in der Methode \lstinline!main! zuerst auf Vorhandensein
eines Integer-Parameters geprüft und im Fehlerfall eine Fehlermeldung geworfen
worauf der Abbruch des Programmes folgt(Zeile 21-23).\\
Bei korrektem Übergabeparameter durchläuft das Programm einen try-Block. In
diesem wird der Parameter auf positives Vorzeichen geprüft und im Fehlerfall
eine Fehlermeldung geworfen worauf der Abbruch des Programmes folgt
(Zeile 26-29).\\
In Zeile 32 gibt das Programm eine Meldung aus, welche den Rückgabewert der
Methode \lstinline!erg!, aufgerufen mit dem Eingabeparameter, seinem kleinsten
ganzzahligen Teiler und dem Quotienten aus beidem, enthält.\\
Bei Erscheinen einer Exception wird diese im catch-Block(Zeile 34-36) gefangen,
der Stack-Trace ausgegeben und das Programm beendet.\\
\lstinline!private static int teiler (int zahl)! liefert den kleinsten
ganzzahligen Teiler von \lstinline!zahl!.\\
\lstinline!private static float erg (int zahl, int a, int b)! liefert die
Quadratwurzel vom Wert des Parameters \lstinline!zahl!.

\end{document}
