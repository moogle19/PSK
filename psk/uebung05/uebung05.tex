\documentclass[a4paper, 12pt]{article}
\usepackage[utf8]{inputenc}
\usepackage{fullpage}
\usepackage[ngerman]{babel}
\usepackage{amsmath}

\usepackage{minted}
\title{5. Übungsblatt zu Programmiersprachenkonzepte}
\author{Manuel Schwarz, Michael Stypa}

\begin{document}

\maketitle

\subsection*{Aufgabe 5.1}
\inputminted{common-lisp}{myreverse.lisp}

\subsection*{Aufgabe 5.2}
\begin{enumerate}
  \item
    \begin{enumerate}
      \item \mintinline[]{common-lisp}
            @(format T "Das Wort heisst ~S." "LISP")@
            \begin{minted}{common-lisp}
Das Wort heisst "LISP".
NIL
            \end{minted}
      \item \mintinline[]{common-lisp}
            @(format T "Das Symbol ist ~&~A." "LISP")@
            \begin{minted}{common-lisp}
Das Wort heisst
LISP.
NIL
            \end{minted}
      \item \mintinline[]{common-lisp}
            @(format T "Die Formatierung ~~S erzeugt diese ~A." 'Ausgabe)@
            \begin{minted}{common-lisp}
Die Formatierung ~S erzeugt diese AUSGABE.
NIL
            \end{minted}
      \item \mintinline[]{common-lisp}
            @(format T "Die Formatierung ~~~S sieht anders aus." A)@ \\
            Bei uninitialisiertem A: Fehler \smallskip \\
            Sonst:
            \begin{minted}{common-lisp}
Die Formatierung ~INHALT sieht anders aus.
NIL
            \end{minted}
    \end{enumerate}
  \item \mintinline[]{common-lisp}
        @(format t "Laenge: ~9,2f~%Breite: ~9,2f@ \\
        \mintinline[]{common-lisp}
        @  ~%Hoehe: ~9,2f~%" "22.34" "134.20" "1.0")@
\end{enumerate}

(format t "Laenge: ~9,2f~%Breite: ~9,2f~%Hoehe: ~9,2f~%" "22.34" "134.20" "1.0")

\subsection*{Aufgabe 5.3}
\inputminted{common-lisp}{durchschnitt.lisp}

\newpage

\subsection*{Aufgabe 5.4}
\inputminted[linenos]{common-lisp}{aufg_4.lisp}

Die Funktion \mintinline[]{common-lisp}@xxx@ bekommt eine Liste übergeben
und prüft, ob zweimal hintereinander das gleiche Element aufgeführt wird.
Ist dies der Fall,
so wird alles (der Rest) nach dem ersten doppelten Element zurückgegeben.
Gibt es so eine Dopplung nicht, so wird NIL zurückgegeben.\\
\\
Beispiel:\\ 
\mintinline[]{common-lisp}@(xxx '(1 2 3 4 5)) = NIL@ \\
\mintinline[]{common-lisp}@(xxx '(1 2 3 3 4 5)) = (4 5)@
\bigskip \\
Funktionalität der \mintinline[]{common-lisp}@do@-Funktion (Iteration):\\
Im ersten Teil wird festgelegt, welche Variablen es gibt, wie ihr Initialwert
lautet und wie sie zu aktualisieren sind.\\
Im zweiten Teil wird eine Abbruchbedingung sowie ein return-Wert festgelegt
und der dritte Teil ist der body der Funktion (der in diesem Fall ebenfalls
einen return-Wert enthält, welcher jedoch bedingt ist).

\end{document}
