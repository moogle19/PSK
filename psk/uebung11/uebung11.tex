\documentclass[a4paper, 12pt]{article}
\usepackage[utf8]{inputenc}
\usepackage{fullpage}
\usepackage[ngerman]{babel}

\usepackage{minted}

\title{11. Übungsblatt zu Programmiersprachenkonzepte}
\author{Manuel Schwarz, Michael Stypa}

\begin{document}

\maketitle

\subsection*{Aufgabe 11.1}
\begin{itemize}
  \item[\emph{Unifikation}]
    Gegeben zwei prädikatenlogische \emph{Aussagen} $A_1 = (X, Y, f(b))$ und
    $A_2 = (a, b, Z)$ mit \emph{Atomen} $a$ und $b$ und Variablen $X$, $Y$ und
    $Z$.\emph{Unifikation} formt Aussagen mit Hilfe der \emph{Substitution} so
    um, dass diese gleich aussehen. Ersetzt man in $A_1$ nun $X$ durch $a$ und
    $Y$ durch $b$ und in $A_2$ $Z$ durch $f(b)$ so erhält man zwei gleiche
    \emph{Aussagen}.
  \item[\emph{Backtracking}]
    Die \emph{Suche} nach einer \emph{Lösung} kann in Einzelschritte unterteilt
    werden. In jedem Schritt gibt es die Möglichkeit mit unterschiedlichen
    Folgeschritten fortzufahren. Stellt man zu einem Zeitpunkt fest, dass der
    Aktuelle Weg nicht zur Lösung führt so geht man die gewählten Schritte
    rückwärts und wählt einen anderen Weg. Dieses Suchverfahren lässt sich
    anschaulich in einer Baum-Struktur darstellen und man bezeichnet es als
    \emph{Backtracking}.
\end{itemize}

\subsection*{Aufgabe 11.2}
\mintinline{prolog}@...@
\inputminted[firstline=58, firstnumber=58, linenos=true]{prolog}{verwandte.pl}

\subsection*{Aufgabe 11.3}
\inputminted[linenos=true]{prolog}{laenge.pl}

\subsection*{Aufgabe 11.4}
\inputminted[linenos=true, firstnumber=12]{prolog}{reise.pl}

\end{document}
