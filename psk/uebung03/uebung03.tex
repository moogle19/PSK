\documentclass[a4paper, 12pt]{article}
\usepackage[utf8]{inputenc}
\usepackage{fullpage}
\usepackage{ngerman}
\usepackage{amsmath}
\usepackage{amssymb}

\usepackage{enumerate}
\usepackage{listings}

\lstset{language = Java}
\lstset{identifierstyle = \ttfamily}
\lstset{showstringspaces = false}
% eig wollt ich nur showspaces ausmachen aber das ist buggy und wird nach
% nem space im string immer "`auf true gesetzt"`

\title{3. Übungsblatt zu Programmiersprachenkonzepte}
\author{Manuel Schwarz, Michael Stypa}

\begin{document}

\maketitle

\subsection*{Aufgabe 3.1}
\subsubsection*{a) Korrigiertes Java-Programm}
\scriptsize
%\lstinputlisting[numbers = left]{aufg1ori.java}
\lstinputlisting[numbers = left]{aufg1cor.java}
\normalsize
\vspace{5mm}

\begin{itemize}
  \item{Z. 3} Sting() args $\rightarrow$ String[] args; wird erkannt (Syntax)
  \item{Z. 5} Anführungszeichen vor der letzten Klammer, wird erkannt (Syntax)
  \item{Z. 16} Die delete()-Methode muss static sein, wird erkannt (Semantisch)
  \item{Z. 18} f.exists() $\rightarrow$ !f.exists(), wird nicht erkannt
  \item{Z. 20} schließende Klammer der if-Anweisung, wird erkannt (Syntax)
  \item{Z. 21} ``:='' $\rightarrow$ ``='', wird erkannt (Lexikalisch)
  \item{Z. 21} Semikolon am Ende der Zeile, wird erkannt (Syntax)
  \item{Z. 22} ``$<$'' $\rightarrow$ ``$>$'', wird nicht erkannt
  \item{Z. 25} int $\rightarrow$ boolean, wird erkannt (Semantisch)
  \item{Z. 29} Die fail()-Methode muss static sein, wird erkannt (Semantisch)
  \item{Z. 32} schließende Klammer der class, wird erkannt (Syntax)
\end{itemize}

\subsubsection*{b) Funktionalität}
Die dem Programm übergebene Datei bzw. das dem Programm übergebene (leere)
Verzeichnis wird gelöscht.
Zunächst wird geprüft, ob genau eine Datei übergeben wurde. Wenn nicht bricht 
das Programm ab, ansonsten wird versucht diese zu löschen.
Dazu wird die delete()-Methode mit dem übergebenen Parameter (Datei oder
Verzeichnis) aufgerufen.
Anschließend wird geprüft, ob die Datei existiert und ob sie evtl.
schreibgeschützt ist. Wenn sie nun nicht existiert oder kein Schreibrecht
besteht, wird eine Exception geworfen.
Falls die eingelesene Datei ein Verzeichnis ist, wird überprüft, ob dieses leer
ist und ansonsten eine Exception geworfen.
Schließlich wird die Datei gelöscht, wobei der Benutzer bei einem Misserfolg
mit Hilfe einer Exception benachrichtigt wird.

\subsection*{Aufgabe 3.2}
\begin{tabular}{|c|c|}\hline
  \textbf{Vorteile}         &\textbf{Nachteile} \\\hline
  Portabilität              & Performance       \\\hline
  Single Stepping debugging &                   \\\hline

  

\end{tabular}

\subsection*{Aufgabe 3.3}
\begin{tabular}{|c|c|}
  \hline
  ZAHL        & Symbol \\ \hline
  3-2-1-meins & Fehler / Symbol warum? \\ \hline
  15,03       & Fehler \\ \hline
  -25         & Integer \\ \hline
  FLOATP      & Symbol \\ \hline
  STRING      & Symbol \\ \hline
  DREI10      & Symbol \\ \hline
  4+1         & Fehler / Symbol warum? \\ \hline
  66/4        & Rational \\ \hline
  -33/11      & Rational \\ \hline
\end{tabular}

\subsection*{Aufgabe 3.4}
\begin{tabular}{|c|c|}
  \hline
  (+ 1 2 3 4 5 6 7 8 9)         & 45 \\ \hline
  (+ -1 (- 3 1))                & 1 \\ \hline
  (- (+ 3 5) (* 2 4) (/ 12 9))  & -4/3 \\ \hline
  (- (+ 3.0 5) (* 2 4) (/ 7 2)) & -3.5 \\ \hline
\end{tabular}

\subsection*{Aufgabe 3.5}
\begin{tabular}{|c|c|}
  \hline
  (ATOM ()) & T \\ \hline
  (ODDP 5) & T \\ \hline
  (SYMBOLP 6) & NIL \\ \hline
  (EQUAL 3 3.0) & NIL \\ \hline
  (NOT (NOT T)) & T \\ \hline
  (NUMBERP (SYMBOLP X)) & NIL \\ \hline
\end{tabular}

\end{document}


