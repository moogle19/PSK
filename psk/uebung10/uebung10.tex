\documentclass[a4paper, 12pt]{article}
\usepackage[utf8]{inputenc}
\usepackage{fullpage}
\usepackage[ngerman]{babel}

\usepackage{tikz}
\usetikzlibrary{shapes,arrows}
\usepackage{xspace}

\title{10. Übungsblatt zu Programmiersprachenkonzepte}
\author{Manuel Schwarz, Michael Stypa}

% TODO dirty
\tikzstyle{lefttoright} = [
  grow = right,
  level 1/.style={sibling distance=1.5cm,level distance=4.5cm},
  level 2/.style={sibling distance=1cm, level distance=4.5cm},
  edge from parent/.style={very thick,draw=blue!40!black!60, shorten >=5pt,
                           shorten <=5pt},
  edge from parent path={(\tikzparentnode.east) -- (\tikzchildnode.west)},
  kant/.style={text width=2cm, text centered, sloped},
  every node/.style={text ragged, inner sep=2mm},
  punkt/.style={rectangle, rounded corners, shade, top color=white,
                bottom color=blue!50!black!20, draw=blue!40!black!60,
                text width=3.5cm, very thick }
]

\begin{document}

\maketitle

\subsection*{Aufgabe 10.1}

\subsection*{Aufgabe 10.2}

Top-Down with Functional Decomposition:\\
\begin{tikzpicture} [lefttoright]
\node[punkt, text width=5.5em] {Lotto-Programm}
  child {
    node[punkt] {Schein eintragen}
    edge from parent
    child {
      node [punkt] {Tip abgeben}
      edge from parent
      child {
        node [punkt] {Nummer erzeugen}
        edge from parent
      }
      child {
        node [punkt] {Regeln prüfen}
        edge from parent
      }
    }
  }
  child {
    node[punkt] {Gewinn-Scheine anzeigen}
    edge from parent
    child {
      node [punkt] {Liste ausgeben}
      edge from parent
    }
    child {
      node [punkt] {Ziehung eingeben}
      edge from parent
    }
  };
\end{tikzpicture}
\bigskip\\
\noindent
Top-Down with data oriented decomposition:\\
\begin{tikzpicture} [lefttoright]
\node[punkt, text width=5.5em] {Lotto-Programm}
  child {
    node[punkt] {Schein-Liste}
    edge from parent
    child {
      node [punkt] {Lotto-Schein}
      edge from parent
      child {
        node [punkt] {Schein-Nummer}
        edge from parent
      }
      child {
        node [punkt] {Schein-Tipp}
        edge from parent
      }
    }
  }
  child {
    node[punkt] {Ziehung}
    edge from parent
    child {
      node[punkt] {Zahlen}
      edge from parent
    }
  }
  child {
    node[punkt] {Regelwerk}
    edge from parent
    child {
      node [punkt] {Gewinnklassen}
      edge from parent
    }
    child {
      node [punkt] {Auszahlungen}
      edge from parent
    }
  };
\end{tikzpicture}
\bigskip\\
\noindent
Class diagram with objectoriented decomposition:\\
\begin{tikzpicture} [lefttoright]
\node[punkt, text width=5.5em] {Lotto-Programm}
  child {
    node[punkt] {Schein-Liste}
    edge from parent
    child {
      node [punkt] {Lotto-Schein}
      edge from parent
      child {
        node [punkt] {Schein-Nummer}
        edge from parent
      }
      child {
        node [punkt] {Schein-Tipp}
        edge from parent
      }
    }
  }
  child {
    node[punkt] {Ziehung}
    edge from parent
    child {
      node[punkt] {Zahlen}
      edge from parent
    }
  }
  child {
    node[punkt] {Regelwerk}
    edge from parent
    child {
      node [punkt] {Gewinnklassen}
      edge from parent
    }
    child {
      node [punkt] {Auszahlungen}
      edge from parent
    }
  };
\end{tikzpicture}
\end{document}
