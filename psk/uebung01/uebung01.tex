\documentclass[a4paper,11pt]{article}
\usepackage[utf8]{inputenc}
\usepackage{fullpage}
\usepackage{ngerman}
\usepackage{amsmath}

\usepackage{enumitem}
\usepackage{listings}

\title{1. Übungsblatt zu Programmiersprachenkonzepte}
\author{Manuel Schwarz, Michael Stypa}

\begin{document}

\maketitle

\subsection*{Aufgabe 1.2}
\begin{enumerate}[label=\alph*)]
  \item Java, C/C++, SQL, PHP, XML, BASH und LUA
  \item VB, Prolog, Ruby und ActionScript
\end{enumerate}

\bigskip

\subsection*{Aufgabe 1.3}
\begin{minipage}{.4\linewidth}
  \begin{enumerate}
    \item Pascal
    \item Fortran
    \item C
    \item Tex
  \end{enumerate}
\end{minipage}
\hspace{.1\linewidth}
\begin{minipage}{.4\linewidth}
  \begin{enumerate}[start=5]
    \item HTML
    \item LISP
    \item Prolog
    \item Cobol
  \end{enumerate}
\end{minipage}

\bigskip

\subsection*{Aufgabe 1.4}
Anstatt bei if klauseln anweisungsblöcke lediglich einzelnanweisungen bei if klauseln?
\begin{lstlisting}
       i = 1;
FOR:   if(i < 10) {
WHILE:     if(sum <= 50) {
               sum = sum +i;
               goto WHILE;
           }
           sum = 0;
           i++;
           goto FOR;
       }
\end{lstlisting}

\end{document}
